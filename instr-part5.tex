\section*{Project V: Specific Aims}

The fifth and final addition to your final project proposal are the statement of three necessary steps toward achieving your overall project aim, called \textit{specific aims}.
The specific aims are the action items of testing your project's central hypothesis. 
They represent specific studies (or papers) you'd plan to execute en route to confirming (or rejecting!) your overall goal. 
The specific aims will be labeled and combined with an expectation (an aim hypothesis) per study. 


Below is a set of specific aims and hypotheses from one of my own awarded grant proposals.


We plan to attain the overall project objective by pursuing the following three specific aims:

\renewcommand{\outlinei}{enumerate}
\begin{outline} 
\1 \textbf{Develop and validate a new experimental approach to correlate soft material chemistry, loading rate, and amplitude with their high-rate behavior and microstructure.} We hypothesize that mechanical testing at both low and high rates and deformation magnitudes with pre- and post-analysis of material networks will underpin our discovery of micromechanical models.
\1 \textbf{Leverage advances in modeling and data assimilation to identify material parameters and the physical models and theory that underpin them.} We hypothesize that data assimilation, integrated into a forward full-resolution multiphase flow model, will describe the fast mechanical behavior of materials with quantifiable uncertainty and inform us when the models need augmentation.
\1 \textbf{Integrate experimental and computational methods to identify soft material behavior with optimal performance.} We hypothesize that by constructing a library of candidate constitutive terms and using our experiments and models, we can identify the physical mechanisms underpinning the observed mechanical response.
\end{outline}


Please note that this portion will not be submitted separately from your final project writeup, but it's a critical part of your project proposal all the same. 
%This is just a placeholder for now
