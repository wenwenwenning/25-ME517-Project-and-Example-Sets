\documentclass{elsarticle}
\input{instr-preamble}
\input{instr-mathsymbols}

\title{Your Paper}
\author{You}

\begin{document}
\begin{center}
    517 Mechanics of Soft Material HW3\\
    26311207 Wen-Ning, Wan
\end{center}
\section{Examples III. Linear Viscoelastic Models}
\subsection{Converting creep to relaxation}
(a) Plot $J_c(t)$
\begin{figure}[h]
    \centering
    \includegraphics[width=0.5\linewidth]{Hw_fig/Fig3_1.png}
\end{figure}
\\
(b) 
\begin{align*}
    \mathcal{L}\{J_c(t)\}&=\mathcal{L}\{\frac{\bar{\varepsilon}}{s\bar{\sigma}}\}\\
    &=\frac{1}{1000}(\frac{10}{s}-\frac{5}{s+\frac{1}{4}}-\frac{3}{s+\frac{1}{8}})\\
    \frac{\bar{\varepsilon}}{\bar{\sigma}}&=\frac{s}{1000}(\frac{10}{s}-\frac{5}{s+\frac{1}{4}}-\frac{3}{s+\frac{1}{8}})\\
    \frac{\bar{\sigma}}{\bar{\varepsilon}}&=\frac{1000}{s}(\frac{s(s+\frac{1}{4})(s+\frac{1}{8})}{10(s+\frac{1}{4})(s+\frac{1}{8})-5s(s+\frac{1}{8})-3s(s+\frac{1}{4})})\\
    \bar{G_r}(s)=\frac{\bar{\sigma}}{s\bar{\varepsilon}}&=\frac{1000}{s}(\frac{s^2+\frac{3}{8}+\frac{1}{32}}{2s^2+\frac{19}{8}s+\frac{10}{32}})\\
    G_r(t)=100&+390.44e^{-1.037t}+9.56e^{-0.151t}
\end{align*}
Creeps happens at $t=4s$, and $t=8s$, while relaxation happens at $t=\frac{1}{1.037}=0.97s$, and $t=\frac{1}{0.151}=6.63s$. Relaxation times are shorter than creep times.
\subsection{Alternate standard linear solid model}
(a)\\
Stress equilibrium and kinematic equilibrium in Kelvin-Voigt model are as follow:
\begin{align*}
    \sigma&=\sigma_k=\sigma_s\\
    \varepsilon&=\varepsilon_k+\varepsilon_s\\
\end{align*}
Consitutive law of spring, and Kelvin-Voigt model
\begin{align*}
        \sigma_s&=E\varepsilon_s\\
    \sigma_k&=E_1\varepsilon_k+\eta\dot{\varepsilon}_k
\end{align*}
Solve the consititutive law:
\begin{align*}
    \sigma&=E_1\varepsilon_k+\eta(\dot{\varepsilon}-\frac{\dot{\sigma}}{E})\\
    \sigma&=E_1(\varepsilon-\frac{\sigma}{E})+\eta(\dot{\varepsilon}-\frac{\dot{\sigma}}{E})\\
    \sigma&=E_1\varepsilon-\frac{E_1}{E}\sigma+\eta\dot{\varepsilon}-\frac{\eta}{E}\dot{\sigma}\\
    (1+\frac{E_1}{E})\sigma&+\frac{\eta}{E}\dot{\sigma}=E_1\varepsilon+\eta\dot{\varepsilon}\\
    \sigma+\frac{\eta}{E+E_1}&\dot{\sigma}=\frac{E_1E}{E_1+E}\varepsilon+\frac{E\eta}{E+E_1}\dot{\varepsilon}
\end{align*}
(b)
Let $a=\frac{\eta}{E+E_1},b=\frac{E_1E}{E_1+E},c=\frac{E\eta}{E+E_1}$
\begin{align*}
    \frac{\bar{\varepsilon}}{s\bar{\sigma}}&=\frac{1+as}{cs^2+bs}\\
    &=\frac{1}{c}(\frac{A}{s}+\frac{B}{s+\frac{b}{c}})\\
    &=\frac{1}{c}(\frac{c}{b}\frac{1}{s}+(a-\frac{c}{b})\frac{1}{s+\frac{b}{c}})
\end{align*}
\begin{align*}
    J_c(t)&=\frac{1}{b}+(\frac{a}{c}-\frac{1}{b})e^{-\frac{b}{c}t}\\
    &=\frac{E+E_1}{EE_1}+(\frac{1}{E}-\frac{E+E_1}{EE_1})e^{-\frac{E_1}{\eta}t}
\end{align*}
\begin{align*}
    \frac{\bar{\sigma}}{s\bar{\varepsilon}}=\frac{b+cs}{s+as^2}=\frac{1}{a}(\frac{ab}{s}+\frac{c-ab}{s+\frac{1}{a}})
\end{align*}
\begin{align*}
    G_r(t)&=\frac{1}{a}(ab+(c-ab)e^{-\frac{1}{a}t})\\
    &=\frac{E_1E}{E_1+E}+(E-\frac{E_1E}{E+E_1})e^{\frac{E+E_1}{\eta}t}
\end{align*}
(c)\\
Coefficient of $J_c(t)$ and $G_r(t)$ are inverse to each other.

\subsection{Frequency response of a 5 term analog model}
(a)
\begin{figure}[h]
    \centering
    \includegraphics[width=0.5\linewidth]{Hw_fig/Fig3_3.jpeg}
\end{figure}
\\
(b)\\
\begin{align*}
G'(\omega)&=G_{\infty}+\sum_k\frac{G_k\omega^2\tau^2_k}{1+\omega^2\tau^2_k}\\
    G'(\omega)&=C_r[10+\frac{200(\omega^2/4)}{1+\omega^2/4}+\frac{100\omega}{1+\omega^2}]\\
    G''(w)&=\sum_k\frac{G_k\omega\tau_k}{1+\omega^2\tau^2_k}\\
    G''(\omega)&=C_r[\frac{100\omega}{1+\omega^2/4}+\frac{100\omega}{1+\omega^2}]
\end{align*}
The loss tangent is:
\begin{align*}
    tan\delta(\omega)=\frac{G''}{G'}
\end{align*}
\begin{figure}[h]
    \centering
    \includegraphics[width=0.5\linewidth]{Hw_fig/Fig3_3_2.png}
\end{figure}
\subsection{Fractional response}
(a) step strain
\begin{figure}[h]
    \centering
    \includegraphics[width=0.5\linewidth]{Hw_fig/Fig3_4_1.png}
\end{figure}
\begin{figure}[h]
    \centering
    \includegraphics[width=0.5\linewidth]{Hw_fig/Fig3_4_1_2.png}
\end{figure}

\newpage

(b) step stress until $t=5$
\begin{equation*}
    \sigma(t)=\sigma_0 H(t)H(5-t)
\end{equation*}
\begin{figure}[h]
    \centering
    \includegraphics[width=0.5\linewidth]{Hw_fig/Fig3_4_2.png}
\end{figure}
\begin{figure}[h]
    \centering
    \includegraphics[width=0.5\linewidth]{Hw_fig/Fig3_4_2_2.png}
\end{figure}
\newpage
(c) consine pulse loading
let
\begin{equation*}
    \sigma(t)=
    \begin{cases}
        \sigma_0\frac{1-cos(2\pi t/T_{load}}{2},\quad0\leq t \leq T_{load}\\
        0,\quad t > T_{load}
    \end{cases}
\end{equation*}
Use superposition:
\begin{equation}
    \sigma_n\approx\sum^n_{j=1}G_{n-j+1}\Delta\varepsilon_j, \quad\Delta\varepsilon_j\equiv\varepsilon_j-\varepsilon_{j-1}
\end{equation}
\begin{equation}
    \Delta\varepsilon_n=\frac{\sigma_n-\sum^n_{j=1}G_{n-j+1}\Delta\varepsilon_j}{G_1},\quad\varepsilon_n=\varepsilon_{n-1}+\Delta\varepsilon_n
\end{equation}
Sample $\alpha$ with logits. Take integers $l\in\{-4,-3,...,3,4\}$
\begin{align*}
    \alpha=logis(l)=\frac{1}{1+e^{-l}}
\end{align*}
\begin{figure}[h]
    \centering
    \includegraphics[width=0.5\linewidth]{Hw_fig/Fig3_4_3.png}
\end{figure}
\begin{figure}[h]
    \centering
    \includegraphics[width=0.5\linewidth]{Hw_fig/Fig3_4_3_2.png}
\end{figure}
\newpage
\subsection{Rheology without a rheometer}
(a)\\
Gravitational potential per unit volume is
\begin{align*}
    \frac{E_{grav}}{V}=\rho gh_0
\end{align*}
and the fraction dissipated in teh first impact is $1-e^2$, where $e$ is the coefficient of restitution.
\begin{align*}
    \frac{E_{diss}}{V}&=\rho g h_0(1-e^2)\\
    &=0.0981\rho(1-e^2)
\end{align*}
if $e =0.5:98.1(1-0.5^2)=73.6J/m^3$
if $e =0.9:98.1(1-0.9^2)=18.6J/m^3$
(b)\\
\begin{align*}
    \sigma(t)&=|E^{*}|Bsin(\omega t+\delta)\\
    y(t)&=\frac{\sigma}{|E^{*}|}=Bsin(\omega t+\delta)=B[sin\omega tcos \delta+cos\omega tsin\delta]=\varepsilon cos\delta+Bsin\delta cos\omega t
\end{align*}
$\varepsilon cos\delta$ is the elastic term
\begin{align*}
    \frac{\omega_{el}}{|E^{*}|}=\int^\varepsilon_0 y_{el}d\varepsilon'=\int^\varepsilon_0 (\varepsilon'cos\delta)d\varepsilon'=\frac{1}{2}\varepsilon^2cos\delta
\end{align*}
Half-cycle occurs at $\varepsilon=B$
\begin{align*}
    \frac{\omega_{el,peak}}{|E^{*}|}=\frac{1}{2}B^2cos\delta\\
    \rightarrow\omega_{el,peak}=\frac{1}{2}|E^{*}|cos\delta B^2
\end{align*}
(c)\\
Half-cycle goes from $0 \rightarrow\pi$
\begin{align*}
    \omega_{diss}=|E^{*}|B^2\int^{\pi}_0sin(\theta+\delta)cos\theta d\theta=|E^{*}|B^2[\frac{\pi}{2}sin\delta]
\end{align*}
(d)\\
\begin{align*}
\frac{\omega_{diss}}{\omega_{el,peak}}=\pi\frac{E''}{E'}=\pi tan\delta
\end{align*}
Approximate $\omega_{diss}$ and $\omega_{el,peak}$
\begin{align*}
    \omega_{diss}&=\rho g(h_0-h)\\
    \omega_{el,peak}&=\rho gh
\end{align*}
\begin{align*}
    tan\delta=\frac{1}{\pi}\frac{\rho g(h_0-h)}{\rho gh}=\frac{2}{\pi}(\frac{h_0}{h}-1)
\end{align*}
(e)\\
\begin{align*}
    t_c&\approx0.025R\\
    f_c&\approx1/t_c\approx40/R
\end{align*}
\begin{align*}
    R&=1m:40Hz\\
    R&=0.01:40kHz
\end{align*}

$\rightarrow f\in[40Hz,40kHz]$
\end{document}