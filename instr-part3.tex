\section*{Project III: Data Gathering and Summary (Due Oct 17)}

Step three of your whitepaper development is the evidence gathering stage. 
Before proposing to investigate something that takes significant time and resources, you must demonstrate that there is a promising lead.  
For this milestone, your task is to \textbf{find, select, and curate} datasets from literature, open datasets, or your own work if available, that are pertinent to your identified gap from the previous step.

Note that you are \textbf{not} expected or required to perform analysis at this stage---that will be the next project checkpoint. 
Instead, focus on actually finding these datasets, understanding what they tell us, and briefly contextualizing these results in relation to both the original study and your own project. 

You'll execute this particular aim as a Powerpoint slide deck of 4-6 slides, saved down as a pdf. \textit{A strong submission will contain all of the following:}


\renewcommand{\outlinei}{itemize}
\begin{enumerate}
\item[\textbf{0.}] \textbf{Title Slide}
\begin{outline}
\1 Project working title, your name, a one-sentence restatement of your research gap. Please use the \hyperlink{https://me.engin.umich.edu/student-intranet/}{ME department template}. 
\end{outline}
\item[\textbf{1.}] \textbf{Brief description of your project}
\begin{outline}
\1 Single sentence summaries of the following:
\2 The state of the field around your topic of interest
\2 The gap in that field
\2 What your project broadly would aim to achieve%A visual summary (table or schematic) listing all data sources (with citation).
\2 General description of the data that you found, and why it's relevant to that aim
\end{outline}
\item[\textbf{2--6.}] \textbf{Individual Dataset Slides}
\begin{outline}
\1 For each selected dataset (aim for 3--5 distinct datasets including both computational and experimental works), include:
\2 Figure/table of data (may be reproduced from literature with citation, or created from open databases/your own data)
\2 Bullet points containing:
\3 Some detail about what the plot of table shows, including context (e.g. how it was generated, under what circumstances, and what variables were controlled-for)
\3 Full citation for the work
\3 Why this dataset is relevant for your proposed project and gap
\end{outline}
\end{enumerate}


%This is just a placeholder for now
