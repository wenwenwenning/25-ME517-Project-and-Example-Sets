\section*{Project Submission: Full $\sim$3pg White Paper + 5 Slides (Due Dec. 15th)}

Your full white paper will consist of the following \textbf{labeled} sections:
\begin{outline}
    \1 \textbf{Overview and Motivation} ($\leq1/2$ page)
        \2 Includes the intellectual merit (why it's worth studying further) and  broader impact (why it'd be beneficial to study).
    \1 \textbf{Technical Background} ($\leq1/2$ page)
        \2 Concise summary of the general field of interest. What is known, what is unknown?
    \1 \textbf{Relationship to Specific/Current Research} ($\leq1/2$ page)
        \2 Summary of the current results and/or techniques and the challenges/issues therein---goal is to set the stage for your own work. 
    \1 \textbf{Research Objectives and Specific Aims}
        \2 State the overall objective of your proposed project and the central hypothesis underpinning the objective.
        \2 State the three specific aims with the corresponding hypotheses.
    \1 \textbf{Expected Outcomes of the Work} ($\leq1/4$ page)
        \2 Describe, briefly, what would be possible if your project were to be completely successful. 
    \1 \textbf{One-Line Budget} (optional/extra credit, but highly encouraged!)
        \2 Estimate the amount of money you would need from a sponsor to make this project happen (assume three years for the project). Note: For FY25, a full time GSRA requires a budgeted amount of \$103,843\footnote{How I arrived at this \href{https://orsp.umich.edu/develop-proposal/budget-and-cost-resources/graduate-student-research-assistant-gsra-cost-estimates}{with link}: \$15,490$\times$2 semesters tuition plus stipend of \$29,192$\times 4/3$, then with 20\% added on that for benefits. Non-tuition costs get an overhead charge of 56\% at U–M, meaning it costs \$1.56 to get \$1 for research. If you need research supplies or want to travel to a conference, you need to request 1.56$\times$ that in your budget.}. 
    \1 \textbf{References} (can be formatted in \footnotesize{footnotesize}\normalsize{-}size font).
\end{outline}

\textbf{Formatting Guidelines:} 
Overall, your whitepaper should be approximately 3 pages, and a maximum of 4 pages including References. 
Please use 11pt font (\textbf{must be} one of Computer Modern/default \LaTeX, Arial, Times, or Palatino Linotype) and margins of 1 inch. 
It is generally good form to have approximately one figure or graphic per page. 

Lastly, you will present your project proposal to your peers as a set of five slides corresponding to the first five points of the whitepaper above. 
This presentation will serve in place of the 5th problem set and officially be on the final exam date of Dec. 15th. 
