\documentclass{elsarticle}
\input{instr-preamble}
\input{instr-mathsymbols}

\title{Your Paper}
\author{You}

\begin{document}
\begin{center}
    517 Mechanics of Soft Material HW3\\
    26311207 Wen-Ning Wan
\end{center}
\section{Examples IV. 3D Time-dependent Behavior Examples}
\subsection*{4--1. Convolutional integrals}

For a cantilever under a uniform load $w$, the elastic tip deflection is
%
\begin{align*}
u_{\max} = \frac{w L^{4}}{8 E I}.
\end{align*}
%
In linear viscoelasticity:
%
\begin{align*}
u_{\max}(t) = \frac{w L^{4}}{8 I}\ J_c(t) = CJ_c(t)
\end{align*}
Let
\begin{align*}
C = \frac{w L^{4}}{8 I}.
\end{align*}


Let $A = J_0 - J_\infty$.  
%
\begin{align*}
0.60/C &= J_\infty + A e^{-30/\tau_c}\\
0.75/C &= J_\infty + A e^{-60/\tau_c}\\
1.00/C &= J_\infty
\end{align*}
%
\begin{align*}
-\frac{0.40}{C} = A e^{-30/\tau_c},
\qquad
-\frac{0.25}{C} = A e^{-60/\tau_c}
\end{align*}

\begin{align*}
\frac{0.25/ C}{0.40/ C} = e^{-30/\tau_c}
\quad\rightarrow\quad
e^{-30/\tau_c} = 0.625
\end{align*}
\begin{align*}
\tau_c = \frac{30}{|\ln(0.625)|} \approx 63.8\ \mathrm{min}.
\end{align*}
\begin{align*}
A = -\frac{0.40}{C}\,\frac{1}{0.625}
  = -\frac{0.64}{C}
\end{align*}
so
\begin{align*}
J_0 = J_\infty + A = \frac{1}{C} - \frac{0.64}{C} 
     = \frac{0.36}{C}.
\end{align*}

\begin{align*}
w = 3\ \mathrm{lb/ft} = 0.25\ \mathrm{lb/in}
\end{align*}
 $L = 48\ \mathrm{in}$.  
\begin{align*}
C = \frac{w L^4}{8I}
  = \frac{0.25 \cdot 48^4}{8}
  = 165{,}888
\end{align*}

Therefore,
\begin{align*}
J_\infty = \frac{1}{C}
         = 6.03\times 10^{-6}\ \mathrm{in^2/lb}
\end{align*}
\begin{align*}
J_0 = \frac{0.36}{C}
     = 2.17\times 10^{-6}\ \mathrm{in^2/lb}
\end{align*}
\begin{align*}
\tau_c \approx 63.8\ \mathrm{min}
\end{align*}

\begin{align*}
\rightarrow
J_c(t)
= 6.0\times 10^{-6}
+ \left(2.2\times 10^{-6} - 6.0\times 10^{-6}\right)
  e^{-t/63.8}
\end{align*}
\subsection*{4--2. Ramp up the torque}
Torque applied to the cylinder is:
\begin{align*}
    T(t) = P_0 a(t) = P_0 \alpha t
\end{align*}

\begin{align*}
    \phi(t)
    = \frac{L}{J_p}\int_0^t J(t-\tau)\,\frac{dT(\tau)}{d\tau}\,d\tau.
\end{align*}

Since $\dfrac{dT(\tau)}{d\tau} = P_0\alpha$ is constant
\begin{align*}
    \phi(t)
    &= \frac{P_0 \alpha L}{J_p}\int_0^t J(t-\tau)\,d\tau \\
\end{align*}
\subsection*{4--3. Ramp up the torque}

(a)\\

\begin{align*}
    \varepsilon(t) = 
    \int_0^t \frac{F(\tau)}{A}\, E_c(t-\tau)\, \mathrm{d}\tau
\end{align*}

\begin{align*}
\rightarrow
    \delta_{\mathrm{ax}}(t)
    = L\,\varepsilon(t)
    = \frac{L}{A}\int_0^t F(\tau)\,E_c(t-\tau)\,\mathrm{d}\tau
\end{align*}

Similarly, the shear stress is
\begin{align*}
    \tau(t) = \frac{F(t)}{A},
\end{align*}
\begin{align*}
   \rightarrow \gamma(t)
    =
    \int_0^t \frac{F(\tau)}{A}\,\mu_c(t-\tau)\,\mathrm{d}\tau
\end{align*}

\begin{align*}
    \rightarrow \delta_{\mathrm{sh}}(t)
    =
    L\,\gamma(t)
    =
    \frac{L}{A}\int_0^t F(\tau)\,\mu_c(t-\tau)\,\mathrm{d}\tau
\end{align*}

\begin{align*}
    \rightarrow \delta_C(t)
    =
    \frac{L}{A}
    \int_0^t F(\tau)\,\big[ E_c(t-\tau) + \mu_c(t-\tau) \big]\,\mathrm{d}\tau
\end{align*}
(b)\\
\begin{align*}
    V(t) = F(t), \qquad 
    M(t) = F(t)\,L, \qquad 
    N(t) \approx 0
\end{align*}

\begin{align*}
    I_y = \frac{h^4}{12}.
\end{align*}

The normal stress:
\begin{align*}
    \sigma_{xx}(t,z)
    =
    -\frac{M(t)\,z}{I_y}.
\end{align*}
At the upper surface ($z = +h/2$),
\begin{align*}
    \sigma_{xx}(t)
    =
    -\frac{6\,F(t)\,L}{h^3}
\end{align*}


The transverse shear stress in a rectangular cross section is zero at the
top and bottom surfaces:
\begin{align*}
    \tau_{xz}(t) = 0.
\end{align*}
\begin{align*}
    \rightarrow\boldsymbol{\sigma}(t)
    =
    \begin{bmatrix}
        -\dfrac{6F(t)L}{h^3} & 0 & 0 \\
        0 & 0 & 0 \\
        0 & 0 & 0
    \end{bmatrix}
\end{align*}

\subsection*{4--4. Ramp up the torque}
(a)\\
Let $Q \in SO(3)$,
\begin{align*}
    \tilde{C} = \tilde{F}^T \tilde{F}
              = (QF)^T (QF)
              = F^T Q^T Q F
              = C.
\end{align*}
Since $I_1(\tilde{C}) = I_1(C)$ and $I_a(\tilde{C},a_0) = I_a(C,a_0)$, the stored energy satisfies
\begin{align*}
    \phi(QF) = \phi(F)
\end{align*}
proving that the potential is frame-indifferent.

(b)\\
\begin{align*}
    G = \{ H \in SO(3) : \phi(FH) = \phi(F)\ \text{for all } F \}
\end{align*}
\begin{align*}
    &\tilde{F} = FH\\
    &\tilde{C} = H^T C H
\end{align*}
The invariant $I_1$ remains unchanged for any $H \in SO(3)$ due to the cyclic property
of the trace. For the fiber invariant,
\begin{align*}
    I_a(FH,a_0)
    = a_0 \cdot (H^T C H) a_0
    = (H a_0) \cdot C (H a_0)
\end{align*}
Requiring $I_a(FH,a_0) = I_a(F,a_0)$ for all symmetric positive-definite $C$
implies that $H a_0 = \pm a_0$. Restricting to the connected component of $SO(3)$
gives the material symmetry group
\begin{align*}
    G = \{\, H \in SO(3) : H a_0 = a_0 \,\}
\end{align*}
i.e.\ all rotations about the fiber direction $a_0$

(c)\\
\begin{align}
    \frac{\partial I_1}{\partial F} &= 2F \\
    \frac{\partial I_a}{\partial F} &= 2(F a_0 \otimes a_0)
\end{align}

\begin{align*}
\rightarrow
    P(F)
    = C_{10} F
    + k (I_a - 1)\, (F a_0 \otimes a_0)
\end{align*}
\begin{align*}
    \sigma = \frac{1}{J} P F^T,
\end{align*}
\begin{align*}
   \rightarrow \sigma(F)
    = \frac{1}{J}
      [
         C_{10}\, b
         + k (I_a - 1)\, (F a_0 \otimes F a_0)
      ]
\end{align*}
\subsection*{4--5. Gent model of rubber elasticity}
(a)\\
Let
\begin{align*}
A = 1 - \frac{I_1 - 3}{J_m}
\end{align*}
\begin{align*}
\phi_{\text{iso}} = -\frac{C_{10}}{2}J_m \ln A
\end{align*}
\begin{align*}
\phi_1
= -\frac{C_{10}}{2}J_m \cdot \frac{1}{A}
\left(-\frac{1}{J_m}\right)
= \frac{C_{10}}{2}\,\frac{1}{A}
\end{align*}
Since
\begin{align*}
A=\frac{J_m-(I_1-3)}{J_m}
\end{align*}
we obtain
\begin{align*}
2\phi_1
=\frac{C_{10}J_m}{J_m-(I_1-3)}
\end{align*}

\begin{align*}
\phi_2
= \frac{\partial}{\partial I_2}
\left[
C_{01}\ln\left(\frac{I_2}{3}\right)
\right]
= \frac{C_{01}}{I_2}
\end{align*}
so that
\begin{align*}
2\phi_2 = \frac{2C_{01}}{I_2}
\end{align*}

Substituting into the general formula,
\begin{align*}
\boldsymbol{\sigma}
= -p\,\mathbf{I}
+ \frac{C_{10}J_m}{J_m-(I_1-3)}\,\mathbf{B}
- \frac{2C_{01}}{I_2}\,\mathbf{B}^{-1}
\end{align*}
(b)\\

Let a incompressible simple shear deformation of magnitude $\gamma$:
\begin{align*}
\mathbf{F} =
\begin{pmatrix}
1 & \gamma & 0\\
0 & 1 & 0\\
0 & 0 & 1
\end{pmatrix},
\qquad
\mathbf{B} = \mathbf{F}\mathbf{F}^T
=
\begin{pmatrix}
1+\gamma^2 & \gamma & 0\\
\gamma & 1 & 0\\
0 & 0 & 1
\end{pmatrix}.
\end{align*}

\begin{align*}
I_1 = \mathrm{tr}\,\mathbf{B} = 3 + \gamma^2,
\qquad
I_2 = 3 + \gamma^2.
\end{align*}

\begin{align*}
\mathbf{B}^{-1} =
\begin{pmatrix}
1 & -\gamma & 0\\
-\gamma & 1+\gamma^2 & 0\\
0 & 0 & 1
\end{pmatrix},
\end{align*}
so
\begin{align*}
B_{12}=\gamma,
\qquad
(B^{-1})_{12}=-\gamma.
\end{align*}

For the Gent-type material with
\begin{align*}
\boldsymbol{\sigma}
= -p\,\mathbf{I}
+ \frac{C_{10}J_m}{J_m-(I_1-3)}\,\mathbf{B}
- \frac{2C_{01}}{I_2}\,\mathbf{B}^{-1},
\end{align*}

\begin{align*}
\sigma_{12}
= \frac{C_{10}J_m}{J_m-\gamma^2}\,B_{12}
- \frac{2C_{01}}{I_2}(B^{-1})_{12}.
\end{align*}
Substituting $B_{12}=\gamma$, $(B^{-1})_{12}=-\gamma$,
$I_1-3=\gamma^2$ and $I_2=3+\gamma^2$ gives
\begin{align*}
\sigma_{12}
=\gamma\left[
\frac{C_{10}J_m}{J_m-\gamma^2}
+ \frac{2C_{01}}{3+\gamma^2}
\right].
\end{align*}

Shear modulus:
Define the shear modulus as the $\gamma$–linear coefficient:
\begin{align*}
\mu(\gamma^2) = \frac{\sigma_{12}}{\gamma}
= \frac{C_{10}J_m}{J_m-\gamma^2}
+ \frac{2C_{01}}{3+\gamma^2}.
\end{align*}

Taking the limit $\gamma\to 0$,
\begin{align*}
\lim_{\gamma\to 0} \mu(\gamma^2)
= \frac{C_{10}J_m}{J_m}
+ \frac{2C_{01}}{3}
= C_{10} + \frac{2}{3}C_{01}.
\end{align*}

\begin{align*}
\displaystyle
\lim_{\gamma\to 0} \mu(\gamma^2)
= C_{10} + \frac{2}{3}C_{01}
\end{align*}



\end{document}